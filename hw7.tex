\documentclass{article}
\usepackage{amsmath,amssymb,amsthm,latexsym,paralist,color}

\theoremstyle{definition}
\newtheorem{problem}{Problem}
\newtheorem*{solution}{Solution}
\newtheorem*{resources}{Resources}

\newcommand{\name}[1]{\noindent\textbf{Name: #1}}
\newcommand{\honor}{\noindent On my honor, as an Aggie, I have neither
  given nor received any unauthorized aid on any portion of the
  academic work included in this assignment. Furthermore, I have
  disclosed all resources (people, books, web sites, etc.) that have
  been used to prepare this homework. \\[1ex]
 \textbf{Signature:} \underline{\hspace*{5cm}} }

% \newcommand{\checklist}{\noindent\textbf{Checklist:}
% \begin{compactitem}[$\Box$] 
% \item Did you add your name? 
% \item Did you disclose all resources that you have used? \\
% (This includes all people, books, websites, etc. that you have consulted)
% \item Did you sign that you followed the Aggie honor code? 
% \item Did you solve all problems? 
% \item Did you submit the pdf file
%   of your homework?
% \item Did you submit a hardcopy of the pdf file in class? 
% \end{compactitem}
% }

\newcommand{\problemset}[1]{\begin{center}\textbf{Problem Set
      #1}\end{center}}
\newcommand{\duedate}[2]{\begin{quote}\textbf{Due dates:} Electronic
    submission of this homework is due on \textbf{#1} on ecampus, a
    signed paper copy of the pdf file is due on \textbf{#2} at the
    beginning of class. \end{quote} }

\newcommand{\N}{\mathbf{N}}
\newcommand{\R}{\mathbf{R}}
\newcommand{\Z}{\mathbf{Z}}

\newcommand{\bl}[1]{\color{blue}#1\color{black}}

\begin{document}
\problemset{7}
\duedate{Tuesday 4/2/2019 before 12:30pm}{4/2/2019}
\name{ Hunter Cleary }
\begin{resources} (All people, books, articles, web pages, etc. that
  have been consulted when producing your answers to this homework)
  \begin{itemize}
      \item http://theanalysisofdata.com/probability/E\_1.html
      \item https://www.mathsisfun.com/data/probability-events-conditional.html
      \item https://sarielhp.org/teach/2004/b/webpage/lec/17\_mincut.pdf
      \item https://en.wikipedia.org/wiki/Karger\%27s\_algorithm
      \item https://web.stanford.edu/class/archive/cs/cs161/cs161.1138/lectures/11/Small11.pdf
  \end{itemize}
  
\end{resources}
\honor

\newpage


\begin{problem}[20 points] 
Suppose that the sample space $\Omega$ is given by the set of positive
integers. Let $\mathcal{F}$ denote the smallest family of subsets of
$\Omega$ such that (a) $\mathcal{F}$ contains all finite sets, (b)
$\mathcal{F}$ is closed under
complements (meaning if $A$ is in $\mathcal{F}$, then $A^c$ is in
$\mathcal{F}$), and (c) $\mathcal{F}$ is closed under countable
unions (so if the sets $E_1, E_2,  \ldots$ are contained in
$\mathcal{F}$, then $\bigcup_{k=1}^\infty E_k$ is contained in
$\mathcal{F}$). 
\begin{compactenum}[(a)]
\item Show that $\mathcal{F}$ is a $\sigma$-algebra. 
\item Prove of disprove: $\mathcal{F}$ is equal to the power set
  $P(\Omega)$. 
\end{compactenum}
\end{problem}
\begin{solution}
\end{solution} \\
\begin{itemize}
    \item (a)
    \begin{itemize}
        \item $\mathcal{F}$ is countable, therefore it contains the empty set.
        \item $\mathcal{F}$ is closed under compliments(For set A there is a set $A^c$)
        \item $\mathcal{F}$ is a countable collection
    \end{itemize}
    $\mathcal{F}$ meets all the conditions of a $\sigma$-algebra set
    \item (b)
    
\end{itemize}

\newpage

\begin{problem}[20 points] 
Suppose that $A$ and $B$ are events in an experiment with $\Pr[ A
\setminus B] = 1/6$, $\Pr[B\setminus A]=1/4$, and $\Pr[ A\cap B] = 1/12$. Find the
probability of each of the following events:
\begin{compactenum}[(a)]
\item $A$,
\item $B$,
\item $A\cup B$,
\item $A^c \cup B^c$. 
\end{compactenum}
\end{problem}
\begin{solution} \\

\begin{itemize}
    \item (a) $Pr[A \backslash B] = Pr[A] - Pr[A \cap B]$\\
    $ - Pr[A] =  - Pr[A \cap B] - Pr[A \backslash B]$\\
    $ Pr[A] =  Pr[A \cap B] + Pr[A \backslash B]$\\
    $ Pr[A] = 1/12 + 1/6$\\
    $ Pr[A] = 3/12$\\
    \\
    $ Pr[A] = 1/4$\\
    
    \item (b)
    $Pr[B \backslash A] = Pr[B] - Pr[A \cap B]$\\
    $ - Pr[B] =  - Pr[A \cap B] - Pr[B \backslash A]$\\
    $ Pr[B] =  Pr[A \cap B] + Pr[B \backslash A]$\\
    $ Pr[B] = 1/12 + 1/4$\\
    $ Pr[B] = 4/12$\\
    \\
     $ Pr[B] = 1/3$\\
     
     \item (c) 
     $Pr[A \cup B] = Pr[A \backslash B] + Pr[A \backslash B] + Pr[A \cap B]$\\
     $Pr[A \cup B] = 1/6 + 1/4 + 1/12$\\
     \\
     $Pr[A \cup B] = 1/2$\\
     
     \item (d) $Pr[A^c \cup B^c] = Pr[A \cap B]^c$ -- DeMorgan's Law \\
     $Pr[A^c \cup B^c] = 1 - Pr[A \cap B]$\\
     $Pr[A^c \cup B^c] = 1 - 1/12$\\
     \\
     $Pr[A^c \cup B^c] = 11/12$\\
     
    
\end{itemize}

\end{solution}

\newpage

\begin{problem}[20 points] 
Give examples of events where (a) $\Pr[A \mid B] < \Pr[A]$, 
(b) $\Pr[A \mid B] =  \Pr[A]$, and (c) $\Pr[A \mid B] > \Pr[A]$. Make
sure that your proofs are complete and self-contained. 
\end{problem}
\begin{solution} \\
\begin{itemize}
    \item (a)\\
    The probability of $Pr[A|B]$ is the probability of A occurring given that B occurs. $(Pr["A$ and $B"]) = P[A] * Pr[B|A]$\\
    \\
    If you were to pick a random card from a shuffled deck of playing cards, your chance for picking an Ace of Spades card is 1 in 52. ($B = 1/52$).\\
    \\
    If A were the chance of picking an Ace card from the deck, $A = 4/52$, then the probability of A occurring after B ($Pr[A|B]$) would be $3/52$.\\
    \\
    $(3/52) < (4/52)$
    
    \item (b)\\
    If B was the probability of drawing a card with a value greater than 7, then its probability will be $6/13$.\\
    \\
    Let A be the probability that the card picked from the deck is a Heart card ($1/4$)\\
    \\
    Since A will happen $1/4$ of the time no matter what the card value is, $Pr[A] = Pr[A|B]$\\
    
    \item (c)\\
    Let there exist three playing cards face down. They are the ace of spades, ace of diamonds, and the ace of hearts.\\
    \\
    Let B be the probability that the ace of spades is drawn ($1/3$)\\
    \\
    Let A be the probability that the ace of hearts is drawn ($1/3$)\\
    \\
    If B occurs, the the probability of A becomes ($1/2$). Making $Pr[A] < Pr[A|B]$ true.\\
    
    
\end{itemize}
\end{solution}

\newpage

\begin{problem}[20 points] 
  There may be several different min-cut sets in a graph. Using the
  analysis of the randomized min-cut algorithm, argue that there can
  be at most $n(n - 1)/2$ distinct min-cut sets.
\end{problem}
\begin{solution} \\
A min-cut is the smallest possible cut that can be made on a graph. Every graph has $2^{n-1}$ cuts, among which at most $n(n - 1)/2$ can be minimum cuts, where n is the number of vertices.\\
\\
Let k be equal to the size of the minimum cut. All vertices will have a degree of k or higher. There are at least $2k/n$ edges. The probability of selecting an edge crossing the cut C in the first step is at most $2/n$. Therefore, the probability of finding a cut that isn't this is $1 - (2/n)$. (Compliment $A^c$ of the original probability A).\\
\\
Beginning at the $m^{th}$ step, there are $n-m+1$ remaining vertices. The graph / multi-graph at this step has $(k/2)*(n-m+1)$ edges. The probability A to select an edge  that crosses cut C is $2/(n-m+1)$.\\
\\
Conditional Probability\\
\\
$Pr[E_{m}|E_{m-1} \cap ...\cap E_{1}] \geq 1 - 2/(n-m+1) = (n-m-1)/(n-m+1)$\\
\\
Summation of conditional probabilities\\
\\
$Pr[E_{m} \cap ...\cap E_{1}] $\\
$=Pr[E_{m}|E_{m-1} \cap ...\cap E_{1}]*Pr[E_{m} \cap ...\cap E_{1}] $\\
$=Pr[E_{m-2}|E_{m-3} \cap ...\cap E_{1}]*Pr[E_{m-1}|E_{m-2} \cap ...\cap E_{1}*Pr[E_{m}|E_{m-1} \cap ...\cap E_{1}]*Pr[E_{m} \cap ...\cap E_{1}]$\\
etc.\\


$$Pr[\cap^{n-2}_{j=1} E_{j}]\geq \prod_{m=1}^{n-2} (n-m-1/n-m+1) = 2/n(n-1)$$

\end{solution}

\newpage

\begin{problem}[20 points] 
The FastCut algorithm by Karger and Stein finds a given minimum cut
  $C$ with probability $P(n) \ge c/\ln n$ for some positive real
  number $c$. How many times should you repeat FastCut so that the probability
  to miss a given minimum cut is less than $1/n$? [Hint: Put the
  formula $1+x\le e^x$ to good use]. 
\end{problem}
\begin{solution} \\
As the graph gets smaller, the probability of making a bad cut choice increase. The algorithm needs to be run more times when the graph is smaller. \\
\\
The algorithm finds a min cut iff: The partial contraction step doesn't contract an edge in the min cut, and at least one of the two remaining contractions does find a min cut.\\
\\
The algorithm returns a min cut with probability $\Omega (1 / log n)$\\
$P(n) \geq ( c/ \ln n )$ for some positive number c\\
If the alogrithm is run $ln^2 n$ times, the probability that all runs fail is 1/n.\\
\\
$$(1-(1/\ln n))^{\ln^2 n} \leq (1/e)^{\ln^2 n} = 1/n $$

\end{solution}









\goodbreak
\checklist
\end{document}
\usepackage[utf8]{inputenc}

\title{CSCE 411 Homework 7}
\author{hncleary }
\date{March 2019}

\begin{document}

\maketitle

\section{Introduction}

\end{document}
