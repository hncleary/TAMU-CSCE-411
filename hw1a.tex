\documentclass{article}
\usepackage[utf8]{inputenc}
\usepackage{graphicx}
\graphicspath{ {./images/} }
\usepackage[margin=1.5in]{geometry}
\usepackage{listings}
\usepackage{amsmath}

\title{CSCE 411 Homework 1A}
\author{Hunter Cleary }
\date{1-21-2019}

\begin{document}

\maketitle

\section{Didgeridoo Breathing}
\paragraph{}
\begin{large}
\\ \\
Didgeridoos were encountered in the lecture while watching the video of sorting algorithms demonstrated on playing cards. Groovy didgeridoo music played in the background. 
\\ \\
\indent A tone can be sustained for an extended amount of time while playing a wind instrument, such as the didgeridoo,  by using a method called circular breathing. This allows for the continuous droning noise of the didgeridoo to remain unbroken. Circular breathing is achieved by breathing in through the nose while simultaneously breathing out through the mouth, maintaining a constant rate of flow. 
\end{large}

\section{Heron's Formula}
\paragraph{}
\begin{center}
\begin{large}
S represents the semi-perimeter of the triangle
\end{large}
\end{center} 
\[ \text{Semi-Perimeter} = \frac{ \text{Side 1} + \text{Side 2} + \text{Side 3} }{2} \]
\[ S = \frac{(A+B+C)}{2} \]
\\ 
\[ A = \sqrt{S(S-A)(S-B)(S-C)} \]
\\

\begin{large}
\\ \\ 
To find the area of a triangle ABC: \\ \\ 
(1) calculate the semi-perimeter ( S ) using AB, BC, & AC \\ \\
(2) substitute S into the original equation
\end{large}



\end{document}
