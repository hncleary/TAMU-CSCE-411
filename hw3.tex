\documentclass{article}
\usepackage{amsmath,amssymb,amsthm,latexsym,paralist}
\usepackage[a4paper, total={6in, 8in}]{geometry}

\theoremstyle{definition}
\newtheorem{problem}{Problem}
\newtheorem*{solution}{Solution}
\newtheorem*{resources}{Resources}

\newcommand{\name}[1]{\noindent\textbf{Name: #1}}
\newcommand{\vertdots}{\underset{\big{\overset{\cdot}{\cdot}}}{\cdot}} 
\newcommand{\diagdots}{_{^{\big\cdot}\cdot _{\big\cdot}}}
\newcommand{\honor}{\noindent On my honor, as an Aggie, I have neither
  given nor received any unauthorized aid on any portion of the
  academic work included in this assignment. Furthermore, I have
  disclosed all resources (people, books, web sites, etc.) that have
  been used to prepare this homework. \\[1ex]
 \textbf{Signature:} \underline{\hspace*{5cm}} }
 

%  \newcommand{\checklist}{\noindent\textbf{Checklist:}
% \begin{compactitem}[$\Box$] 
% \item Did you add your name? 
% \item Did you disclose all resources that you have used? \\
% (This includes all people, books, websites, etc. that you have consulted)
% \item Did you sign that you followed the Aggie honor code? 
% \item Did you solve all problems? 
% \item Did you submit (a) the pdf file of your homework?
% \item Did you submit (b) a hardcopy of the pdf file in class? 
% \end{compactitem}
% }



\newcommand{\problemset}[1]{\begin{center}\textbf{Problem Set
      #1}\end{center}}
\newcommand{\duedate}[2]{\begin{quote}\textbf{Due dates:} Electronic
    submission of the pdf file of this homework is due on
    \textbf{#1} on ecampus, a signed paper copy of the pdf file is due
    on \textbf{#2} at the beginning of class. \end{quote} }

\newcommand{\N}{\mathbf{N}}
\newcommand{\R}{\mathbf{R}}
\newcommand{\Z}{\mathbf{Z}}


\begin{document}
\problemset{3}
\duedate{2/14/2019 before noon}{2/14/2019}
\name{ Hunter Cleary }
\begin{resources} (All people, books, articles, web pages, etc. that
  have been consulted when producing your answers to this homework)
  \begin{enumerate}
    \item Lecture Powerpoint Slides
      \item http://numbers.computation.free.fr/Constants/Algorithms/fft.html
      \item http://www.pi314.net/eng/multipli\_fft.php
      \item https://arxiv.org/abs/1305.4745
      \item https://groups.csail.mit.edu/netmit/sFFT/
      \item https://dsp.stackexchange.com/questions/2317/what-is-the-sparse-fourier-transform
      \item https://en.wikipedia.org/wiki/Partition\_of\_a\_set
  \end{enumerate}
\end{resources}
\honor

\newpage
% Make sure that you describe all solutions in your own words. Read chapters 30 and 16 in our textbook. 

\begin{problem} 
\begin{compactenum}[(a)]
\item (10 points) Suppose that you are given a polynomial 
$$ A(x) = \sum_{k=0}^{n-1} a_k x^k.$$ 
The input to the FFT of length $n$ is given by an array containing the coefficients
$(a_0,\ldots, a_{n-1})$. Describe the output of the FFT in terms of
the polynomial $A(x)$. 
\\
\item (15 points) Let $\omega$ be a primitive $n$th root of unity. 
The fast Fourier transform implements the multiplication with
  the matrix 
$$ F = (\omega^{ij})_{i,j\in [0..n-1]}.$$
Show that the inverse of the $F$ is given by 
$$ F^{-1} = \frac{1}{n}  (\omega^{-jk})_{j,k\in [0..n-1]}$$
% [Hint: $x^n-1= (x-1)(x^{n-1}+\cdots + x + 1),$ so any power
% $\omega^\ell\neq 1$  must be a root of $x^{n-1}+\cdots + x + 1$.  ]  
% Thus, the inverse FFT, called IFFT, is nothing but the FFT using
% $\omega^{-1}$ instead of $\omega$, and multiplying the result with
% $1/n$. 
\item (10 points) Describe how to do a polynomial multiplication using the FFT and
  IFFT for polynomials $A(x)$ and $B(x)$ of degree $\le n-1$. Make
  sure that you describe the length of the FFT and IFFT needed for
  this task.  \\
\item (15 points) How can you modify the polynomial multiplication algorithm based
  on FFT and IFFT to do multiplication of long integers in base 10?
  Make sure that you take care of carries in a proper way. \\
\end{compactenum}
\end{problem}
\begin{solution} \\
(a)\\
% (a) The fast fourier transform algorithm computes the DFT, Discrete Fourier Transform, (or inverse IDFT) of a sequence. \\
Coefficient representation:\\
$A(x) = a_0 +a_1x + \cdots + A_{n-a}x^{n-1}$\\ 
\\
Point Value representation:\\
Understood as the function $x \xrightarrow[]{} y = A(x)$\\
$A(x): \{(x_0,y_0),(x_1,y_1),\dots, (x_{n-1},y_{y-1})\}$\\ 
\\
The conversion from coefficient representation to point value representation:\\
$$
\begin{vmatrix}
y_0\\
y_1\\
\cdots\\
y_{n-1}\\
\end{vmatrix}
=
\begin{vmatrix}
1&x_0&x_0^2&\cdots&x_0^{n-1}\\
1&x_1&x_1^2&\cdots&x_1^{n-1}\\
\cdots&\cdots&\cdots&\ddots&\cdots\\
1&x_{n-1}&x_{n-1}^2&\cdots&x_{n-1}^{n-1}\\
\end{vmatrix}
\begin{vmatrix}
a_0\\
a_1\\
\cdots\\
a_{n-1}\\
\end{vmatrix}
$$\\
The FFT make the run time of this process much faster, hence the name.\\
The polynomial is divided into its even and odd powers:\\
$A(x) = A_{even}(x^2) + xA_{odd}0(x^2)$\\
$ A_{even}(x^2) & xA_{odd}0(x^2)$ are then evaluated at the $1/2n^{th}$ roots of unity.\\
$A(\omega^k) = A_{even}(v^k) + \omega^kA_{odd}(V^k), 0 \leq k \leq n/2$\\
$A(\omega^{k+n/2}) = A_{even}(v^k) - \omega^kA_{odd}(V^k), 0 \leq k \leq n/2$\\
\xrightarrow{}Recursion\\
\\
(b) ?\\
\paragraph{}
(c)\\
Point value representation multiplication runs in O(n) time\\
\\ To perform polynomial multiplication on A(x) and B(x):\\
$$A(x) = \sum_{n=0}^{n-1} a_kx^k$$
$$B(x) = \sum_{n=0}^{n-1} b_kx^k$$
Compute the Fourier transform of A(x) and B(x)\\
\\
Compute C(x) as the product of A(x) and B(x) in point value form.\\
$C(x) = A(x)*B(x)$\\
Compute the inverse Fourier transform of $C(x)$.\\
$$C(x) = \sum_{n=0}^{2n-1} c_kx^k$$\\
\\
\\
(d)\\
Polynomial multiplication through FFT could be use on base 10 integers by generating polynomials from these values:\\
$123456789 = 1*10^8 + 2*10^7 + 3*10^6 + 4*10^5 + 5*10^4 + 6*10^3 + 7*10^2 + 8*10^1 + 9*10^0 $\\
Using the same method as presented in polynomial multiplication:\\
$$A(x) = \sum_{n=0}^{n-1} a_k10^k$$
$$B(x) = \sum_{n=0}^{n-1} b_k10^k$$
$$C(x) = \sum_{n=0}^{2n-1} c_k10^k$$\\
It would functionally work the same, with the final summation resulting in an integer rather than a polynomial.\\

\end{solution}

\newpage 

\begin{problem} (20 points) You overhear a conversation where someone
  mentions that Morgenstern proved an $\Omega(n\log n)$ lower bound on
  the fast Fourier transform and someone else mentions that a group of
  MIT researchers found in 2012 a faster than fast Fouier transform
  that is $o(n\log n)$. These two comments seem to contradict each
  other. Do your research and find out what Morgenstern really proved
  and under what circumstances the MIT algorithm can improve on the
  FFT.
\end{problem}
\begin{solution} \\
Morgensten's proof (1973): \\
When constants used in the computation are bounded, the lower bound for any FFT is $\Omega(n log n)$. This only applies to the unnormalized Fourier transform.\\
\\
MIT algorithm:\\
If a signal has a small number of k non-zero Fourier coefficients then the output can be represented using only k coefficients. (Sparse Fourier Transform) This makes the complexity potentially lower if the signal, in a length domain of N, there are a number of outputs k $<<$ N tha are non-zero. The other N - k outputs become negligible. Because N-k results are zero, they do not need to be generated. \\
\\
$\Omega(klogn)$ for an exactly k sparse case.\\
$O(klognlo(n/k))$ for general case.\\
$\Omega(klog(n/k) / loglogn)$ for sample complexity.\\



\end{solution}

\newpage

\begin{problem} Solve the following two problems on matroids.
\begin{compactenum}
\item (10 points) Solve exercise 16.4-3 on page 443 of [CRLS]. 
\item (20 points)
Solve exercise 16.4-4 on page 443 of [CLRS]. 
\end{compactenum}
\end{problem}
\begin{solution} \\
\\
(1) Show that if (S,I) is a matroid , then (S,I') is matroid.\\
\\
To prove (S,I') is a matroid:
\begin{enumerate}
    \item S is finite
    \item I' is hereditary
    \item (S,I') satisfies the exchange property
\end{enumerate}
It is assumed that S is finite\\
\\
I' is a finite non-empty set.\\
The maximal of I is A.\\
$A \subseteq S - (S-A')$\\
There must exist some other subset B within I' such that $S-B \supseteq S-A \subseteq A'$\\
$B \in I'$\\
If B exists, $A \in I'$ and $|B| < |A|$, then and element x can be found in $B-A$ such that $A \cup \{x\} \in I'$\\
There exists some $y \in x - (S-B)$ such that $(S-B)\cup \{y\} $ is a maximal set it I.\\
$B\cup \{x\}$ is independent in I'\\
\\
\\
(2)\\
Properties of Matroid:\\
For (S,I)\\
\begin{enumerate}
    \item S must be finite
    \item Hereditary: Set I contains at least one value and is from the subsets of S so that $B \in I, \\  A\subseteq B ,\  \& \ A \in I$
    \item Exchange Property: $A \in I, B \in I $ and $|A|<|B|$ for $x \in B-1$\\
    such that $A\cup \{X\} \in 1$
\end{enumerate}
$(S,I)$\\
$I = \{A  \ : \ |A\cap S_i| \leq 1$ for $i = 1,2,\cdots,k\}$\\
$S_i$ is forming the partition of S\\
S is a non-empty set because the partitions of S are non-empty. In the partition of a set, each element is grouped into non-empty subsets that uniquely contains only one of the elements. \\
It is assumed that I is non-empty. Because each of the partitions holds only one of the unique elements, it is hereditary.(The subset of every subset is independent.)\\
If $A \in I, B \in I $ and $|A|<|B|$ , B must have at least one element more than A. This element must be from a partition of S that is not in A. I fulfills this property.\\




















































































































$



\end{solution}


% Discussions on ecampus are always encouraged, especially to clarify
% concepts that were introduced in the lecture. However, discussions of
% homework problems on ecampus should not contain spoilers. It is okay to
% ask for clarifications concerning homework questions if needed. 
\medskip



\goodbreak
\checklist
\end{document}
