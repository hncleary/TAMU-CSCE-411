\documentclass{article}
\usepackage{amsmath,amssymb,amsthm,latexsym,paralist}
\usepackage{graphicx}
\usepackage{verbatim}
\usepackage[a4paper, total={6in, 8in}]{geometry}
\usepackage{mathtools}
\DeclarePairedDelimiter{\ceil}{\lceil}{\rceil}

\theoremstyle{definition}
\newtheorem{problem}{Problem}
\newtheorem*{solution}{Solution}
\newtheorem*{resources}{Resources}

\newcommand{\name}[1]{\noindent\textbf{Name: #1}}
\newcommand{\honor}{\noindent On my honor, as an Aggie, I have neither
  given nor received any unauthorized aid on any portion of the
  academic work included in this assignment. Furthermore, I have
  disclosed all resources (people, books, web sites, etc.) that have
  been used to prepare this homework. \\[1ex]
 \textbf{Signature:} \underline{\hspace*{5cm}} }

%  \newcommand{\checklist}{\noindent\textbf{Checklist:}
% \begin{compactitem}[$\Box$] 
% \item Did you add your name? 
% \item Did you disclose all resources that you have used? \\
% (This includes all people, books, websites, etc. that you have consulted)
% \item Did you sign that you followed the Aggie honor code? 
% \item Did you solve all problems? 
% \item Did you submit (a) the pdf file of your homework?
% \item Did you submit (b) a hardcopy of the pdf file in class? 
% \end{compactitem}
% }



\newcommand{\problemset}[1]{\begin{center}\textbf{Problem Set
      #1}\end{center}}
\newcommand{\duedate}[2]{\begin{quote}\textbf{Due dates:} Electronic
    submission of the pdf file of this homework is due on
    \textbf{#1} on ecampus, a signed paper copy of the pdf file is due
    on \textbf{#2} at the beginning of class. \end{quote} }

\newcommand{\N}{\mathbf{N}}
\newcommand{\R}{\mathbf{R}}
\newcommand{\Z}{\mathbf{Z}}


\begin{document}
\problemset{5}
\duedate{2/28/2019 before noon}{2/28/2019}
\name{ Hunter Cleary }
\begin{resources} (All people, books, articles, web pages, etc. that
  have been consulted when producing your answers to this homework)
    \begin{itemize}
        \item Lecture Slide
        \item http://www.csanimated.com/animation.php?t=Aggregate\_analysis
        \item http://www.cs.cornell.edu/courses/cs3110/2011fa/supplemental/lec20-amortized/amortized.htm
    \end{itemize}
\end{resources}
\honor

\newpage
Make sure that you describe all solutions in your own words. Read
chapter 17 in our textbook. 

\begin{problem}[10 points]
Exercise 17.1-1 on page 456. \\
\\
If the set of stack operations included a MULTIPUSH operation, which pushes k items onto the stack, would the $O(1)$ bound on the amortized cost of stack operations continue to hold?
\end{problem}
\begin{solution} \\
\\
The $O(1)$ bound on the amortized cost does not hold. The cost of a combination of MULTIPUSH or MULITPOP would be $O(n)$ each. In a worst-case scenario k would be a large value, causing the cost of the set of operations to far exceed $O(1)$. \\ \\ \\ 
\end{solution}



\begin{problem}[10 points]
Exercise 17.1-2 on page 456. \\
\\
Show that if a DECREMENT operation were included in the k-bit counter example, n operations would cost as much as $\Theta(nk)$ time.
\end{problem}
\begin{solution} \\
\\
If an array of k bits contains all ones, the INCREMENT operation must flip all k bits. $\Theta(k)$\\
\\
If an array of k bits contains all zeroes except for the left-most bit, the DECREMENT operation must flip all k bits. $\Theta(k)$\\
\\
Since either operation could cost $\Theta(k)$ , the cost of n operations could be $\Theta(nk)$
\\
\end{solution}

\newpage

\begin{problem}[20 points]
Exercise 17.1-3 on page 456.\\
\\
Suppose we perform a sequence of n operations on a data structure in which the i th operation costs i if i is an exact power of 2, and 1 otherwise. Use aggregate analysis to determine the amortized cost per operation.
\end{problem}

\begin{solution} \\
\\



\begin{tabular}{lll}
Operations & Cost & Sum  \\
1          & 1    &    1         \\
2          & 2    & 3        \\
3          & 1    &   4          \\
4          & 4    &   8          \\
5          & 1    &   9          \\
6          & 1    &   10          \\
7          & 1    &     11        \\
8          & 8    &       19      \\
9          & 1    &         20    \\
10         & 1    &           21  \\
11         & 1    &             22\\
12         & 1    & 23            \\
13         & 1    &   24          \\
14         & 1    &     25        \\
15         & 1    &       26      \\
16         & 16   &  42 \\
\end{tabular}\\
\\
\\
Total$ = n + 2^1 + 2^2 + 2^3\cdots$ \\
\\
$\ceil{42/16} = 3$\\ 
\\
$T(n) = n + \sum_{i=0}^{\ceil{logn}} 2^{i}$\\
\\
$O(3n) = O( n )$\\
\\
$T(n)/n = O(1) $
\\

\end{solution}

\newpage

\begin{problem}[20 points]
Exercise 17.2-2 on page 459.
\end{problem}
\begin{solution} \\
\\
Assumes amortized costs for a given operation.\\
\\
Can be achieved by :\\
Cost of 2 if i is an exact power of 2\\
Cost of 3 if is not an exact power of 2\\
\\
$T(16) \leq 3*2 + 13*3 = 45 $\\
$42 \leq 45$\\
\\
$ 3(2^n -1) - (n + 1)\leqT(n) = n + \sum_{i=0}^{\ceil{logn}} 2^{i}$   \\
\\
With these values the the cost will always be less than actual.
\\ \\ \\
\end{solution}


\begin{problem}[20 points]
Exercise 17.3-1 on page 462.
\end{problem}
\begin{solution} \\
\\
Amortized costs using $\Phi'$ are the same as the costs using $\Phi$\\
\\
Amortized Cost: $c_k' = c_k + \Phi(D_k) - \Phi(D_{k-1})$\\
\\
$\Phi'(D_k) = \Phi(D_k) - \Phi(D_{0})$ for all $i \geq 0$\\
$\Phi'(D_0) = \Phi(D_0) - \Phi(D_{0}) = 0$\\
$\Phi'(D_k) = \Phi(D_0) - \Phi(D_{0})  \geq 0$\\
\\
$c_k' = c_k + \Phi'(D_k) - \Phi'(D_{k-1})$\\
$c_k' = c_k + (\Phi(D_0) - \Phi(D_{0})) - ( \Phi(D_0) - \Phi(D_{0}) )$\\
$c_k' = c_k + \Phi(D_k) - \Phi(D_{k-1})$\\
\therefore\\
$c_k' = c_k$\\


\end{solution}

\newpage

\begin{problem}[20 points]
Exercise 17.3-4 on page 462.
\end{problem}
\begin{solution} \\
Potential function $\Phi$ on stack is the number of objects on the stack.\\
$\Phi(D_0) = 0$\\
$\Phi(D_i) \geq 0$\\
\\
Textbook:\\
Amortized cost of PUSH operation\\
$\Phi(D_i) - \Phi(D_{i-1}) = ( s + 1 ) - s$ \\
$=c_i + \Phi(D_i) - \Phi(D_{i-1}) $\\
$=1+1$\\
$=2$\\
MULTIPOP\\
$\Phi(D_i) - \Phi(D_{i-1}) = k'$\\
$=c_i + \Phi(D_i) - \Phi(D_{i-1}) $\\
$=k' + k'$\\
$=0$\\
\\
$c_k = c_k + \Phi(D_k) - \Phi(D_{k-1})$\\
\\
Cost for n operations:\\
$\sum_{i=1}^{n} c_k = \sum_{i=1}^{n} c_k + \Phi(D_k) - \Phi(D_{k-1})$\\
From 0 to n objects\\
$\sum_{i=1}^{n} c_k = \sum_{i=1}^{n} c_k + \Phi(D_0) - \Phi(D_{n})$\\
Stack objects potential:\\
$\sum_{i=1}^{n} c_k = \sum_{i=1}^{n} c_k + s_0 - s_n$
\\
PUSH = 2 , POP \& MULTIPOP = 0\\
$\therefore$\\
The total cost is bounded by $\leq 2n + s_0 - s_n$\\
\\



\end{solution}


% Discussions on ecampus are always encouraged, especially to clarify
% concepts that were introduced in the lecture. However, discussions of
% homework problems on ecampus should not contain spoilers. It is okay to
% ask for clarifications concerning homework questions if needed. Make
% sure that you write the solutions in your own words. 


\medskip



\goodbreak
\checklist
\end{document}
