\documentclass{article}
\usepackage{amsmath,amssymb,amsthm,latexsym,paralist}
\usepackage[a4paper, total={6in, 8in}]{geometry}


\theoremstyle{definition}
\newtheorem{problem}{Problem}
\newtheorem*{solution}{Solution}
\newtheorem*{resources}{Resources}

\newcommand{\name}[1]{\noindent\textbf{Name: #1}}
\newcommand{\honor}{\noindent On my honor, as an Aggie, I have neither
  given nor received any unauthorized aid on any portion of the
  academic work included in this assignment. Furthermore, I have
  disclosed all resources (people, books, web sites, etc.) that have
  been used to prepare this homework. \\[1ex]
 \textbf{Signature:} \underline{\hspace*{5cm}} }

%  \newcommand{\checklist}{\noindent\textbf{Checklist:}
% \begin{compactitem}[$\Box$] 
% \item Did you add your name? 
% \item Did you disclose all resources that you have used? \\
% (This includes all people, books, websites, etc. that you have consulted)
% \item Did you sign that you followed the Aggie honor code? 
% \item Did you solve all problems? 
% \item Did you submit (a) the pdf file of your homework?
% \item Did you submit (b) a hardcopy of the pdf file in class? 
% \end{compactitem}
% }



\newcommand{\problemset}[1]{\begin{center}\textbf{Problem Set
      #1}\end{center}}
\newcommand{\duedate}[2]{\begin{quote}\textbf{Due dates:} Electronic
    submission of the pdf file of this homework is due on
    \textbf{#1} on ecampus, a signed paper copy of the pdf file is due
    on \textbf{#2} at the beginning of class. \end{quote} }

\newcommand{\N}{\mathbf{N}}
\newcommand{\R}{\mathbf{R}}
\newcommand{\Z}{\mathbf{Z}}


\begin{document}
\problemset{2}
\duedate{2/7/2019 before 11:00am}{2/7/2019}
\name{Hunter Cleary}
\begin{resources} (All people, books, articles, web pages, etc. that
  have been consulted when producing your answers to this homework)
  \begin{enumerate}
      \item http://faculty.simpson.edu/lydia.sinapova/www/cmsc250/LN250\_Weiss/L14-RecRel.htm
      \item https://brilliant.org/wiki/insertion/
      \item How to: Prove by Induction - Proof of a Recurrence Relationship
      \\https://www.youtube.com/watch?v=UThZ\_AfqEek
      \item https://brilliant.org/wiki/karatsuba-algorithm/
      \item https://courses.csail.mit.edu/6.006/spring11/exams/notes3-karatsuba

  \end{enumerate}
\end{resources}
\honor

\newpage
% \noindent Make sure that you describe all solutions in your own words. \\[1ex]
% Read chapters 2 and 4 in our textbook before attempting to solve these
% problems. 

\medskip
\noindent\textbf{Problem A.} Solve the following subproblems of
Problem A. 


\begin{problem}[15 points]
Exercise 2.3-3 on page 39 in [CLRS]. 
\end{problem}
\begin{solution} \\
\\
Proof by Induction
\begin{enumerate}

    \item Prove that statement is true for $k = 2$  $(n = 2^2)$
    \item Assume the statement is true for $n = 2^k$
    \item Show that the statement is true for $n = 2^{k+1}$ is true
\end{enumerate}
(1)\\
$T(n) = n\log{n}$\\
$T(2) = 2$\\
$2\log{2} = 2$\\
$2*1 = 2$\\
\\ 
(2)\\
Assume $2T(n/2) + n$ if $n = 2^k$ is true.\\
\\
(3)\\
When n = $2^{k+1}$\\
$n\log{n} = 2T(2^{k+1} /2) + 2^{k+1}$\\
$= 2T(2^{k+1-1}) + 2^{k+1}$\\
$= 2T(2^{k}) + 2^{k+1}$\\
Since $n\log{n} = T(n)$ for $2^k$ is assumed\\
$= 2(2^k\log{2^k}) + 2^{k+1}$\\
$= 2(k*2^k) + 2^{k+1}$\\
$= 2*k*2^k + 2^{k+1}$\\
$= k*2^{k+1} + 2^{k+1}$\\
$= (k+1)2^{k+1}$\\
$= 2^{k+1}\log{2^{k+1}}$\\
$\therefore$\\
$T(n)=n\log{n}$ 
\end{solution}

\begin{problem}[15 points]
Exercise 2.3-4 on page 39 in [CLRS]. 
\end{problem}
\begin{solution} \\
Insertion Sort Recurrence\\
\\
$T(n) = $
\begin{cases}
$ 1$ &\mbox{if } n = 1 \\ 
$ T(n-1) + (n-1) $ &\mbox{if } n > 1 \\ 
\end{cases}\\ \\ \\
$ T(n) = O (n^2) $ 

\end{solution}

\newpage

\begin{problem}[20 points]
Karatsuba's algorithm to multiply large integers was sketched in the
lecture. (a) Explain how the method works with a small example (multiply
$11001100_2 \times 11110111_2$). (b) Derive pseudocode for
Karatsuba's algorithm. 
\end{problem}
\begin{solution} \\
Karastuba's algorithm multiplies two numbers using divide and conquer. \\ \\
The algorithm uses 3 subproblems.\\
\\
\\
For $x*y$\\
$x_h and x_l$ are the leftmost half and right-most half bits respectively \\
b represents the base the number is represented in\\
$a = x_h*y_h$\\
\\
$d = x_l*y_l$\\
\\
$e = (x_h + x_l)(y_h + y_l)- a - d $\\
\\
Then $(x*y) = ab^n + eb^{n/2} + d$
Where each multiplication is recursed using the same steps.\\
\\
(Example)\\
\\
To multiply $204 * 247$ $(11001100 * 11110111)$\\
\\
$a_1 = 20 *24 $ , $d_1 = 4 * 7 $ , $ e_1 = (20+4)(24+7) - a_1 - d_1$\\
\\
$a_1 = 20 * 24 $\\
\indent     $a_2 = 2 * 2 $\\
\indent     $d_2 = 0 * 4$\\
\indent     $e_2 = (20 + 24)(0 + 4) - 4 - 0 = 172$\\
\indent     $20*24 = 4*10^2 + 8*10 + 0$\\
\indent     $a_1 = 480$
\\
\\
$d_1 = 4 * 7  = 28$\\
\\
$e_1 = (24)(31) - 480 - 28$\\
\indent    $a_2 = 2 * 3 $\\
\indent     $d_2 = 4 * 1$\\
 \indent    $e_2 = (2 + 4)(3 + 1) - 6 - 4 = 8$\\
\indent     $24*31 = 6*10^2 + 14*10 + 4 = 744 $\\
$e_1 = 744 - 480 - 28$\\
$e_1 = 236$\\
\\
$204*247 = 480*10^2 + 236 *10 + 28$\\
$204*247=50388$
\\
\newpage
Karatsuba Psuedo Code
\begin{verbatim}
    function multiply(x,y){
        b = ~base of number representation
        n = number of digits
        xLeftHalf = digits of x from 1 to n/2
        xRightHalf = digits of x from n/2 to n
        yLeftHalf = digits of y from 1 to n/2
        yRightHalf = digits of y from n/2 to n
        
        a = multiply(xLeftHalf, yLeftHalf)
        d = multiply(xRightHalf, yRightHalf)
        e = multiply ((xLeftHalf + xRightHalf)(yLeftHalf + yRightHalf)) - a - b
        
        return a*b^n + e*b^(n/2) + d
    }
\end{verbatim}


\end{solution}

\newpage

\begin{problem}[15 points]
Exercise 4.2-5 on page 82 in [CLRS]. 
\end{problem}
\begin{solution} \\
\\
68 x 68 Matrix \\
$T(n) = 132464T(n/68)+n^2$\\
$T(n) = \Theta (n^{log_{68}{132464}})$\\
$T(n) = \Theta (n^{2.795128487})$\\
\\
70 x 70 Matrix \\
$T(n) = 143640T(n/70)+n^2$\\
$T(n) = \Theta (n^{log_{70}{143640}})$\\
$T(n) = \Theta (n^{2.79512269})$\\
\\
72 x 72 Matrix \\
$T(n) = 155424T(n/72)+n^2$\\
$T(n) = \Theta (n^{log_{72}{155424}})$\\
$T(n) = \Theta (n^{2.795147391})$\\
\\
The 68 x 68 matrix has the most efficient run-time with $T(n) = \Theta (n^{2.795128487})$\\
\\
Strassen's algorithm is $\Theta(n^{2.81})$ , which makes the theoretical discovered method faster.
\end{solution}

\newpage

\begin{problem}[20 points]
Exercise 4.2-7 on page 83 in [CLRS]
\end{problem}
\begin{solution} \\
\\
a + bi\\
c + di\\
\\
Product of Numbers \\
$(a+bi)(c+di)$ =\\ 
$= ac +adi + bic + bidi$\\
$= ac + adi + bci - bd$\\
$= ac - bd + (ad+bc)i$\\
\\
\begin{verbatim}
    function complexMultiply(a,b,c,d){
        part1 = (a+b)*(c+d)  //= ac - bd + (ad+bc)i
        part2 = a*c   
        part3 = b*d    
        ac - bd = part2 - part3                     //real
        ad + bc = part1 - part2 + part3             //imaginary
        
    }
\end{verbatim}

\end{solution}

\newpage

\begin{problem}[15 points]
Exercise 4.5-3 on page 97 in [CLRS] 
\end{problem}
\begin{solution} \\
\\
Show that the solution to binary-search reccurrence is \\
\\
$T(n) = T(n/2) + \Theta(1) = \Theta(\log{n})$\\
\\
--Master Theorem--\\
\\
$a=1\\ b=2\\ f(n) = \Theta(1)$\\
\\
Matches Case 2:\\
$f(n) = \Theta(n^{\log_b{a}})$\\
$= \Theta(n^{\log_2{1}})$\\
$= \Theta(n^{0})$\\
$= \Theta(1)$\\
\\
\therefore\\
\\
$T(n) = \Theta(n^{\log_b{a}} * \log{n})$\\
$= \Theta(n^{\log_2{1}} * \log{n})$\\
$= \Theta(1 * \log{n})$\\
$= \Theta(\log{n})$\\
\\
$T(n) = \Theta(\log{n})$\\


\end{solution}



% Work out your own solutions, unless you want to risk an honors
% violation!
\medskip



\goodbreak
\checklist
\end{document}
