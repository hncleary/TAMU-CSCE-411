\documentclass{article}
\usepackage{amsmath,amssymb,amsthm,latexsym,paralist}

\theoremstyle{definition}
\newtheorem{problem}{Problem}
\newtheorem*{solution}{Solution}
\newtheorem*{resources}{Resources}

\newcommand{\name}[1]{\noindent\textbf{Name: #1}}
\newcommand{\honor}{\noindent On my honor, as an Aggie, I have neither
  given nor received any unauthorized aid on any portion of the
  academic work included in this assignment. Furthermore, I have
  disclosed all resources (people, books, web sites, etc.) that have
  been used to prepare this homework. \\[1ex]
 \textbf{Signature:} \underline{\hspace*{5cm}} }
 
% \newcommand{\checklist}{\noindent\textbf{Checklist:}
% \begin{compactitem}[$\Box$] 
% \item Did you add your name? 
% \item Did you disclose all resources that you have used? \\
% (This includes all people, books, websites, etc. that you have consulted)
% \item Did you sign that you followed the Aggie honor code? 
% \item Did you solve all problems? 
% \item Did you submit the pdf file resulting from your latex file 
%   of your homework?
% \item Did you submit a hardcopy of the pdf file in class? 
% \end{compactitem}
% }

\newcommand{\problemset}[1]{\begin{center}\textbf{Problem Set #1}\end{center}}
\newcommand{\duedate}[2]{\begin{quote}\textbf{Due dates:} Electronic
    submission of this homework is due on \textbf{#1} on ecampus, a
    signed paper copy of the pdf file is due on \textbf{#2} at the
    beginning of class. \end{quote} }

\newcommand{\N}{\mathbf{N}}
\newcommand{\R}{\mathbf{R}}
\newcommand{\Z}{\mathbf{Z}}


\begin{document}
\problemset{8}
\duedate{4/18/2019 before 12:30pm}{4/18/2019}
\name{ Hunter Cleary }
\begin{resources} (All people, books, articles, web pages, etc. that
  have been consulted when producing your answers to this homework)
  \begin{itemize}
      \item https://en.wikipedia.org/wiki/Conjunctive\_normal\_form#Converting\_from\_\\first-order\_logic
      \item https://www8.cs.umu.se/kurser/TDBA77/VT06/algorithms/BOOK/BOO\\K4/NODE157.HTM
      \item http://mathworld.wolfram.com/IsomorphicGraphs.html
      \item http://www.cs.tau.ac.il/~ronitt/COURSES/F08/lec7.pdf
      \item https://en.wikipedia.org/wiki/Co-NP
      \item https://www.geeksforgeeks.org/partition-problem-dp-18/
      \item https://en.wikipedia.org/wiki/Partition\_problem
      \item https://en.wikipedia.org/wiki/Subset\_sum\_problem
  \end{itemize}
\end{resources}
\honor

\newpage

Read Chapter 34 in our textbook. 

\begin{problem} (20 points) % 
A boolean formula is said to be in disjunctive normal form if and only
if it is the disjunction of clauses, where each clause is the
conjunction of literals (e.g. $(x \wedge \neg y\wedge \neg z) \vee (\neg
x \wedge \neg y \wedge z)$ is in disjunctive normal form). 
Show that there exists a polynomial-time algorithm to determine
whether a boolean formula in disjunctive normal form is satisfiable. 
\end{problem}
\begin{solution} \\
\\
A boolean formula of disjunctive normal form is satisfiable if it can be evaluated to True / 1 / T.\\
\\
If a clause exists containing a variable x such that ($\neg x \wedge x$), it cannot be evaluated to True.\\
\\
Where C is a set of clauses in a DNF boolean formula\\
\\
For every clause l[i] in the set of i clauses C \{ \\
\indent \ \ \ \ if ( for variable x there exists $(\neg x \wedge x)$ )\{\\
\indent \indent \ \ \ \ return false //not satisfiable \\
\indent \indent \}\\
\indent \}\\
\indent return true //satisfiable\\


\end{solution}

\newpage

\begin{problem} (20 Point) 
Dr. S.M. Art Aleck thinks he deserves the Turing award as he gave the
following compelling argument that SAT can be solved in polynomial
time. Given a boolean formula $f$ in conjuctive normal form, simply
convert $f$ to disjunctive normal form and use the poly-time algorithm
from the previous problem to determine whether $f$ is
satisfiable. Explain why Dr. Aleck is mistaken. 
\end{problem}
\begin{solution} \\
% \\
% Example of disjunctive normal form notation\\
% $(x \wedge \neg y\wedge \neg z) \vee (\neg x \wedge \neg y \wedge z)$\\
% Converted to conjunctive normal form\\
\\
You cannot "simply convert" conjunctive form to disjunctive normal form.\\
\\
Converting from CNF to DNF while preserving equivalence is an NP-hard problem. This means that it is as least as hard as the hardest problems in NP. Finding a polynomial algorithm to solve any NP-hard problem would give polynomial algorithms for all problems in NP.\\

\end{solution}

\newpage

\begin{problem} (20 points) 
Consider the language 
$$\text{GRAPH-ISOMORPHISM} = \{ (G_1, G_2) : G_1 \text{
  and } G_2 \text{ are isomorphic
graphs}\}.$$ Prove that GRAPH-ISOMORPHISM is in NP by \textbf{describing} a
polynomial-time algorithm to verify the language. 
\end{problem}
\begin{solution} \\
\\
Two graphs which contain the same number of graph vertices connected in the same way are isomorphic.\\
\\
Given an input x which is two graphs $G_1$ and $G_2$, let there be a certificate c that refers to the indices of the graph and marks the mapping of the vertices between the two graphs.\\
\\
It must be verified that for all pairs of vertices that are adjacent in $G_1$, they can be mapped to a pair of correlating vertices in the graph $G_2$.\\
\\
A polynomial time algorithm A given input graphs x and certificate c must verify that the input graphs are isomorphic.\\
\\
If the certificate is a permutation of the vertices, perform the permutation on graph $G_1$ and check if it is equal to graph $G_2$.\\
\\

\end{solution}

 \newpage

\begin{problem} (20 points) % If NP != co-NP then NP != P. 
Exercise 34.2-10 on page 1066. [Hint: Read Chapter 34.2 and make sure you
understand the definition of co-NP.] \\
\\
Prove that if NP != co-NP, then P != NP.

\end{problem}
\begin{solution} \\
\\
A decision problem X is a member of co-NP if and only if its complement $\Bar{X}$ is in the complexity class NP.\\
\\
You can prove the above by showing that P = NP implies NP = co-NP. Assuming P = NP, let a decision problem $X \in $ co-NP . $\Bar{X} \in NP$, but since NP = P, it can be concluded that $\bar{X} \in P$.\\
\\
$X = \Bar{X} \in P$, $\therefore$ co-NP $\subseteq P$. \\
$P \subseteq $ co-NP, $\therefore$ P = co-NP\\
$P = NP$, $\therefore$ NP = co-NP\\
\\
Since P = NP implies NP = co-NP, NP != co-NP implying P != NP is true.\\

\end{solution}

\newpage

\begin{problem} (20 points) % Set partition is NP complete
Exercise 34.5-5 on page 1101 [Hint: Reduce SUBSET SUM
to SET PARTITION.] \\
\\
The \textbf{set-partition problem} takes as input a set S of numbers. The question is whether the numbers can be partitioned into two sets A and $\bar{A} = S - A$ such that $\sum_{x \in A} x = \sum_{x \in \bar{A}} x$. Show that the set partition problem is NP-complete.

\end{problem}
\begin{solution} \\
\\
To show that a problem is NP-Complete:
\begin{enumerate}
    \item there is a P time algorithm that solves A
    \item any NP-Complete problem B can be reduced to A
    \item the reduction of B to A runs in P time
    \item the original problem A has a solution iff B has a solution
\end{enumerate}
\\
Set Partition $\in$ NP: guessing the two partitions, then using a certifier to verify that the two have equal sums\\
\\
Reduction:\\
\\
This problem is solved by reducing \textbf{subset-sum}.\\
\\
Let X be a set of integers with target $t$. Let $s$ be the sum of elements that exist in X. Let there be two partitions that have equal sums. Verifying this set partition means the that the problem $\in$ NP.\\
\\
The reduction from the sub-set problem to the set-partition problem will take P time.\\
\\
The set X bleongs to the subset sum if it belongs to the set-partition problem. $X' = X \cup \{s-2t\}$\\
\\
If there are numbers in the set partition problem that sum to $t$, the remaining numbers in X sum to $s-t$. There exists a partition of $X'$ into two where each partition sums to $s-t$. Given this, one of the sets contains the number $s-2t$. After removing this number from the set, there exists a set of numbers that sums to $t$. All of those numbers exist in X.\\
\\
\end{solution}

% Make sure that you write the solutions in your own words!
\medskip



\goodbreak
\checklist
\end{document}
